\documentclass[12pt]{report}

% package to activate greek language 
\usepackage[greek,english]{babel}
\usepackage{indentfirst}

% DOCUMENT LAYOUT
\usepackage{geometry} 
\geometry{a4paper, textwidth=5.5in, textheight=8.5in, marginparsep=7pt, marginparwidth=.6in}
\setlength\parindent{8mm}
\setlength\parskip{5mm}

% FONTS
\usepackage{xltxtra}
\usepackage[table]{xcolor}
\defaultfontfeatures{Mapping=tex-text}
\usepackage{libertine}

% package to handle graphics
\usepackage{graphicx}
% package to handle multiple figures in a minipage
\usepackage{subfigure}
%package to activate XeTeX font manager
\usepackage{fontspec}

% PDF SETUP
% ---- FILL IN HERE THE DOC TITLE AND AUTHOR
\usepackage[driverfallback=dvipdfmx, bookmarks, colorlinks, breaklinks, pdftitle={Efi Kaltirimidou},pdfauthor={efikalti}]{hyperref}  
\hypersetup{linkcolor=blue,citecolor=blue,filecolor=black,urlcolor=blue} 

% change captions especially for greek language - if the document is in ENGLISH, they should vanish
\addto\captionsenglish{%
  \renewcommand\prefacename{Πρόλογος}%
  \renewcommand\refname{Αναφορές}%
  \renewcommand\abstractname{Περίληψη}%
  \renewcommand\bibname{Βιβλιογραφία}%
  \renewcommand\chaptername{Κεφάλαιο}%
  \renewcommand\appendixname{Παράρτημα}%
  \renewcommand\contentsname{Περιεχόμενα}%
  \renewcommand\listfigurename{Κατάλογος Σχημάτων}%
  \renewcommand\listtablename{Κατάλογος Πινάκων}%
  \renewcommand\indexname{Ευρετήριο}%
  \renewcommand\figurename{Σχήμα}%
  \renewcommand\tablename{Πίνακας}%
  \renewcommand\partname{Μέρος}%
  \renewcommand\enclname{Συνημμένα}%
  \renewcommand\ccname{Κοινοποίηση}%
  \renewcommand\headtoname{Προς}%
  \renewcommand\pagename{Σελίδα}%
  \renewcommand\seename{βλέπε}%
  \renewcommand\alsoname{βλέπε επίσης}%
  \renewcommand\proofname{Απόδειξη}%
  }

%%%%%%%%% END OF PREAMBLE %%%%%%%%%%%%

\begin{document}

% make title out of \author, \title, \date  specified in the preamble
\begin{titlepage}
%
%
\begin{figure}[ht]
\begin{minipage}{0.15\textwidth}
\vspace{-10mm}
\includegraphics[width=20mm]{../images/aristotle_logo.png}
\end{minipage}
%
\begin{minipage}{0.70\textwidth}
{
ΑΡΙΣΤΟΤΕΛΕΙΟ ΠΑΝΕΠΙΣΤΗΜΙΟ\\[0.5\baselineskip]
ΘΕΣΣΑΛΟΝΙΚΗΣ\\[0.5\baselineskip]
Τμήμα Πληροφορικής
}
\end{minipage}
%
\end{figure}

\vspace{10mm}
\hbox{ % Horizontal box
\hspace*{0.08\textwidth} % Whitespace to the left of the title page
\rule{1.5pt}{0.80\textheight} % Vertical line
\hspace*{0.02\textwidth} % Whitespace between the vertical line and title page text
\parbox[b]{0.90\textwidth}{ % Paragraph box which restricts text to less than the width of the page


{\large \textsc{Πτυχιακή Εργασία}}\\[\baselineskip] % Tagline or further description
{\noindent\LARGE\bfseries
Σχεδίαση και Υλοποίηση Web
Πλατφόρμας για την Υποβολή Επιστημονικών Εφαρμογών σε Κατανεμημένα Συστήματα
}\\[2\baselineskip] % Title

{\Large Ευφροσύνη Καλτιριμίδου}\\[3\baselineskip] % Author name

{\em Επιβλέπων:}\\[0.25\baselineskip]
{\large Αθηνά Βακάλη}\\[0.25\baselineskip]
{\large Καθηγήτρια}

\vspace{0.15\textheight} % Whitespace between the title block and the publisher
{\noindent Θεσσαλονίκη 2017}\\[\baselineskip] % Publisher and logo
}}
%
%
\end{titlepage}

\include{blank_page}


% Create table of contents
\tableofcontents
\include{blank_page}

% Intro
\input{thesis_intro}


% Frontend
% print no page number
\thispagestyle{empty}

\chapter{Front - End Architecture}

To frontend είναι το πιο σημαντικό κομμάτι της εφαρμογής για τον χρήστη. Αν η εφαρμογή ήταν ένα σπίτι, το frontend θα ήταν η εξωτερική πρόσοψη του σπιτού και η οικοδεσπότης, ο οποίος καλωσορίζει τους ανθρώπους μέσα.  Το frontend είναι το κομμάτι της εφαρμογής που χρησιμοποιεί ο χρήστης για να αποκτήσει, να μεταβάλλει και να αποθηκεύσει δεδομένα. Στην συγκεκριμένη εργασία, το frontend είναι αποκλειστικά μια web εφαρμογή.
\newline
Όλη η επίδραση με τον χρήστη γίνετε μέσω του frontend, γιαυτό η σχεδίαση του αποτελεί το δεύτερο βήμα στην υλοποίηση της εφαρμογής. Η τεχνική υλοποίηση του θα γίνει μετά την τεχνική υλοποίηση του backend καθώς η λειτουργικότητα του στηρίζετε στα API calls που προσφέρει το backend.

% leave 60mm empty space below
\vspace{60mm}

\section{Στόχοι της σχεδίασης}
Οι αρχές που ακολουθεί η σχεδίαση του frontend χωρίζετε σε δύο ειδών, τις γενικές αρχές και τις αρχές ασφάλειας.
\begin{description}
\item [Γενικές αρχές] Αναφέρονται σε όλο το frontend και στοχεύουν να κάνουν την εφαρμογή απλή και εύκολη για τον χρήστη.
\newline
Μερικές γενικές αρχές:


\begin{itemize}
  \item Σε κάθε σελίδα του frontend, θα πρέπει να είναι ξεκάθαρες οι διαθέσιμες ενέργειες του χρήστη.
  \item Σε οποιαδήποτε σελίδα βρίσκετε ο χρήστης θα πρέπει να μπορεί να πάει σε οποιαδήποτε άλλη σελίδα της εφαρμογής με μία μόνο κίνηση.
   \item Για κάθε ενέργεια του χρήστη θα πρέπει να υπάρχει μια άμεση απάντηση της εφαρμογής, είτε με αλλαγή περιεχομένου, είτε με ενημερωτικό μύνημα κ.τ.λ.
   \item Σε περίπτωση αποτυχίας κάποιας ενέργειας, θα πρέπει να ενημερώνετε με λεπτομέρεια ο χρήστης ώστε να γνωρίζει πως να συνεχίσει. Αυτό περιλαμβάνει την χρήση μηνυμάτων που θα εμφανίζονται στην οθόνη, το focus της σελίδας σε συγκεκριμένα πεδία ή δεδομένα κ.τ.λ.
\end{itemize}


\item [Αρχές ασφάλειας] Οι αρχές ασφάλειας αναφέρονται σε συγκεκριμένα κομμάτια του frontend και είναι κυρίως κομμάτι της ασφάλειας των δεδομένων.

\begin{itemize}
  \item Η εφαρμογή θα χρησιμοποιεί πρωτόκολλα ασφάλειας για την μεταφορά των δεδομένων του χρήστη.
  \item Ο χρήστης θα παραμένει συνδεδεμένος για πεπερασμένο χρονικό διάστημα και στη συνέχεια θα πρέπει να ανανεώσει την σύνδεση του.
\end{itemize}

\end{description}

\section{Τεχνικά χαρακτηριστικά}

Οι τεχνολογίες που χρησιμοποιούνται σε κάθε εφαρμογή ποικίλουν ανάλογα με τα άτομα που την υλοποίησαν, τα χαρακτηριστικά και τις δυνατότητες της εφαρμογής.
\newline
Για τις web εφαρμογές, δύο τεχνολογίες αποτελούν αναπόσπαστα κομμάτια τους:

\begin{description}
\item [Hypertext Markup Language (\textbf{HTML})] Αποτελεί την γλώσσα σήμανσης που δημιουργεί τις σελίδες της εφαρμογής.

\item [Cascading Style Sheets (\textbf{CSS})] Η οποία περιγράφει την μορφοποίηση του κάθε αντικειμένου στη σελίδα. 
\end{description}


Σε συνδυασμό με μία client-based γλώσσα, όπως η Javascript, αποτελούν τα συστατικά στοιχεία κάθε web εφαρμογή σήμερα.

\begin{description}
\item [JavaScript (\textbf{JS})] Είναι μια client-based γλώσσα προγραμματισμού, η οποία δίνει δυνατότητα για δυναμικές ιστοσελίδες και εφαρμογές στο διαδίκτυο.
\end{description}

Για την υλοποίηση της εφαρμογής χρησιμοποιήθηκε το web framework \textbf{AngularJS}, ένα δομικό framework για δυναμικές web εφαρμογές. Χρησιμοποιεί την HTML ως γλώσσα προτύπου και επεκτείνει τη σύνταξη της για να εκφράσει σαφώς και συνοπτικά τα στοιχεία της εφαρμογής. Διάφορες  λειτουργίες της AngularJS μειώνουν ένα μεγάλο μέρος του κώδικα που θα έπρεπε να γραφτεί. Βασίζετε στη Javascript, και την επεκτείνει για να προσθέσει επιπλέον λειτουργίες προς διευκόλυνση του προγραμματιστή.

\subsection{Εκδόσεις}

Η εφαρμογή έχει υλοποιηθεί και δοκιμαστεί, στο κομμάτι του frontend με τις παρακάτω εκδόσεις:

\begin{description}
\item [HTML] 5
\item [CSS] 3
\item [JavaScript] 5
\item [AngularJS] 1.6.*
\end{description}


% leave 60mm empty space below
\vspace{60mm}

\section{Λειτουργίες Frontent}

Στο frontend της εφαρμογής έχουν πρόσβαση όλοι οι χρήστες της  \textit{Ιδρυματικής Συστοιχίας ΑΠΘ} με τα στοιχεία πιστοποίησης του λογαριασμού τους στο \textit{GRID}. Έχουν πρόσβαση στις παρακάτω λειτουργίες:

\begin{description}
  \item[$\bullet$ Σύνδεση] O χρήστης θα παρέχει το username και password που έχει πάρει κατά την εγγραφή του ως μέλος της \textit{Ιδρυματικής Συστοιχίας ΑΠΘ}.\textit{Προσοχή} , η εφαρμογή δεν παρέχει μέθοδο εγγραφής. Η εφαρμογή θα στείλει τα δεδομένα με ασφαλή πρωτόκολλα στο backend για επιβεβαίωση. Αν τα δεδομένα δεν είναι σωστά θα εμφανίσει το ανάλογο μήνυμα λάθους αλλιώς η εφαρμογή θα μεταβεί στο dashboard του χρήστη.
  \item[$\bullet$ Έλεγχος εργασιών] Ο χρήστης μπορεί να δει τις εργασίες που έχει κάνει submit στον cluster, σε τι κατάσταση βρίσκονται και να τις φιλτράρει βάση κατάστασης, ημερομηνίας υποβολής και ονόματος.
    \item[$\bullet$ Υποβολή εργασίας] Σε αυτή την λειτουργία ο χρήστης μπορεί να επιλέξει να υποβάλει το εκτελέσιμο της εργασίας από τον τοπικό του δίσκο ή να δημιουργήσει η εφαρμογή ένα εκτελέσιμο για αυτόν με βάση παραμέτρους που θα επιλέξει. Επίσης μπορεί να υποβάλει αρχεία εισόδου από τον τοπικό του δίσκο ή να δώσει το path που βρίσκονται μέσα στη συστοιχία.
    \item[$\bullet$ Παραλαβή αποτελεσμάτων] Αν κάποια εργασία έχει εκτελεστεί επιτυχώς, μπορεί να επιλέξει να κατεβάσει τα αποτέλεσμα της στον τοπικό του δίσκο μέσω της εφαρμογής.
\end{description}

\section{Σενάρια χρήσης - Use cases}

\subsection{Είσοδος - Login}

Οι χρήστες της \textit{Ιδρυματικής Συστοιχίας ΑΠΘ} μπορούν να συνδεθούν στην εφαρμογή με τα στοιχεία που χρησιμοποιούν και για τις υπόλοιπες επιστημονικές υπηρεσίες που προσφέρει η συστοιχία. Η πρώτη σελίδα της εφαρμογής είναι η σελίδα του \textbf{Login}. Παρακάτω φαίνεται το σενάριο χρήσης για το \textbf{Login} καθώς και μια εικόνα της σελίδας από την εφαρμογή.

\begin{figure}[t]
\caption{Use case - Login}
\includegraphics[width=16cm]{../images/login-use-case.png}
\centering
\end{figure}
\clearpage

\begin{figure}[t]
\caption{Screenshot - Login}
\includegraphics[width=16cm]{../images/login-screenshot.png}
\centering
\end{figure}
\clearpage

Μετά την επιτυχή επιβεβαίωση των στοιχείων, ο χρήστης μεταφέρετε στην κεντρική σελίδα της εφαρμογής.

\begin{figure}[t]
\caption{Screenshot - Home Page}
\includegraphics[width=16cm]{../images/home-page-screenshot.png}
\centering
\end{figure}
\clearpage


Αν τα στοιχεία του χρήστη είναι λανθασμένα θα του εμφανίσει ανάλογο μήνυμα και θα κοκκινίσει τα πεδία εισαγωγής.

\begin{figure}[t]
\caption{Screenshot - Invalid credentials}
\includegraphics[width=16cm]{../images/invalid-credentials-screenshot.png}
\centering
\end{figure}
\clearpage

Σε περίπτωση που ο χρήστης έχει ξεχάσει τον κωδικό του, πατώντας τον σύνδεσμο \textit{Forgot you password} θα ανοίξει ένα καινούργιο παράθυρο στον \textit{browser} το οποίο αναλύει την διαδικασία ανάκτησης κωδικού, στο \textit{Grid Wiki}.

\begin{figure}[t]
\caption{Use case - Forgot password}
\includegraphics[width=16cm]{../images/forgot-password-case.png}
\centering
\end{figure}
\clearpage


\subsection{Έλεγχος εργασιών - View jobs}

Από το navigation menu στα δεξιά, της κάθε σελίδας, οι χρήστες μπορούν να μεταφερθούν στην σελίδα προβολής των εργασιών που έχουν υποβάλλει στη συστοιχία.

Αρχικά βλέπουν όλες τις εργασίες που έχουν υποβάλλει, και στη συνέχεια μπορούν να φιλτράρουν τις εργασίες για να δουν όποιες τους ενδιαφέρουν.
\newline
Το φιλτράρισμα μπορεί να γίνει με βάση τρεις διαφορετικούς παράγοντες. 
\newline
Την κατάσταση της εργασίας, η οποία μπορεί να είναι:
\begin{itemize}
\item Running - Η εργασία εκτελείται
\item Finished - Η εργασία έχει ολοκληρωθεί επιτυχώς
\item Stopped - Η εργασία έχει διακόψει την εκτέλεση της πριν ολοκληρωθεί
\end{itemize}
Την ημερομηνία υποβολής της εργασίας, από την πιο πρόσφατη έως την πιο παλιά και αντίστροφα.
\newline
Και τέλος αλφαβητικά με βάση το όνομα της εργασίας.
\newline
Τα φίλτρα μπορούν να χρησιμοποιηθούν και συνδυαστικά για να βγάλουν πιο συγκεκριμένα αποτελέσματα.

\begin{figure}[bp!]
\caption{Use case - View jobs}
\includegraphics[width=16cm]{../images/view-jobs-case.png}
\centering
\end{figure}
\clearpage

\begin{figure}[bp!]
\caption{Screenshot - View Jobs}
\includegraphics[width=16cm]{../images/view-jobs-screenshot.png}
\centering
\end{figure}
\clearpage


\subsection{Υποβολή νέας εργασίας - Submit job}

Από το αντίστοιχο μενού \textit{Navigation > Jobs} στα δεξιά της σελίδας, ο χρήστης μπορεί να επιλέξει το \textit{Submit job} για να υποβάλει μια νέα εργασία.  
\newline
Για να γίνει η υποβολή ο χρήστης θα πρέπει απαραίτητα να υποβάλει ένα εκτελέσιμο αρχείο. 
Επίσης μπορεί να προσθέσει αρχεία εισόδου για το πρόγραμμα ή να δώσει  \textit{file path} στο οποίο είναι αποθηκευμένα τα αρχεία εισόδου μέσα στην ιδρυματική συστοιχία. Τα αρχεία εισόδου που θα υποβάλει μέσα από το \textit{web application} μπορούν να είναι περισσότερα από ένα, καθώς και ολόκληροι φάκελοι. 
\newline

Στο κάτω μέρος της σελίδας μπορεί να δει ποια αρχεία έχει δηλώσει για υποβολή και να αφαιρέσει οποιοδήποτε δεν θέλει να συμπεριληφθούν στην εργασία.
\newline

\begin{figure}[bp!]
\caption{Use case - Create new job}
\includegraphics[width=16cm]{../images/create-job-case.png}
\centering
\end{figure}
\clearpage

\begin{figure}[bp!]
\caption{Screenshot - Create new job}
\includegraphics[width=16cm]{../images/submit-job-screenshot.png}
\centering
\end{figure}
\clearpage


Τέλος, αφού έχει υποβληθεί ένα εκτελέσιμο αρχείο, ο χρήστης μπορεί να υποβάλει την εργασία για εκτέλεση πατώντας το κουμπί \textit{submit}.

\begin{figure}[t]
\caption{Use case - Submit job}
\includegraphics[width=16cm]{../images/submit-job-case.png}
\centering
\end{figure}
\clearpage





\include{blank_page}


% API
% print no page number
\thispagestyle{empty}

\chapter{Service layer - Restful API}

\par
Τα κομμάτια του frontend και του backend, παρόλο που είναι ανεξάρτητα το ένα από το άλλο, για την σωστή λειτουργία της εφαρμογής είναι απαραίτητο να επικοινωνήσουν μεταξύ τους. Την επικοινωνία τους αναλαμβάνει ένα τρίτο κομμάτι της εφαρμογής που ονομάζετε \textbf{Service Layer}.\par Σε πολλές εφαρμογές, το \textbf{Service Layer} υλοποιήτε από το κομμάτι του backend καθώς οι ίδιες μέθοδοι που κάνουν την επικοινωνία backend με frontend κάνουν και την επικοινωνία backend με τη βάση δεδομένων.Και σε αυτήν την περίπτωση το \textbf{Service Layer} έχει υλοποιηθεί μαζί με το backend.\par Σε γενικές γραμμές, πρόκειται για ένα σύνολο σαφώς καθορισμένων μεθόδων επικοινωνίας το οποίο για συντομία το λέμε \textbf{API (application programming interface)}.Το API ορίζει τον σωστό τρόπο για έναν προγραμματιστή να γράψει ένα πρόγραμμα που ζητά υπηρεσίες από λειτουργικό σύστημα (OS) ή άλλη εφαρμογή. Τα API υλοποιούνται με λειτουργικές κλήσεις που αποτελούνται από ρήματα και ουσιαστικά. Η απαιτούμενη σύνταξη περιγράφεται συνήθως στο documentation της κάθε εφαρμογής.


% leave 60mm empty space below
\vspace{60mm}

\section{Περιγραφή}
	Τα API αποτελούνται από δύο στοιχεία. Το πρώτο είναι ένα σύνολο κανόνων που περιγράφει τον τρόπο ανταλλαγής πληροφοριών μεταξύ προγραμμάτων, που γίνεται με τη μορφή αίτησης για επεξεργασία και επιστροφής των απαραίτητων δεδομένων. Το δεύτερο είναι το λογισμικού γραμμένο με τις προδιαγραφές που ορίστηκαν από τους κανόνες και δημοσιεύεται για χρήση.

Το λογισμικό που επιθυμεί να αποκτήσει πρόσβαση στις δυνατότητες και τις δυνατότητες του API λέγεται ότι καλεί το API και το λογισμικό που υλοποιεί το API λέγεται ότι το δημοσιεύει.

\section{Σχεδίαση του API}
	Ο σωστός σχεδιασμός του API βασίζετε στις αρχές της διαφάνειας και της ασφάλειας των δεδομένων. Το API θα πρέπει να αποκρύπτει τις μεθόδους και την πολυπλοκότητα της επικοινωνίας από και προς τις δύο υπηρεσίες.Έτσι, ο σχεδιασμός του API επιχειρεί να παρέχει μόνο τα εργαλεία που θα περίμενε να παραλάβει το frontend.
\newline

\section{Είδη APIs}
Υπάρχουν τριών ειδών APIs. 

\begin{description}
\item[Local APIs] Eίναι η αρχική μορφή, από την οποία ήρθε το όνομα. Προσφέρουν υπηρεσίες OS ή middleware σε προγράμματα εφαρμογών. Το API της Microsoft .NET, το TAPI (API τηλεφωνίας) για φωνητικές εφαρμογές και τα API πρόσβασης σε βάσεις δεδομένων είναι  Local API. 
\item[Web APIs] Αυτά τα API έχουν σχεδιαστεί για να αντιπροσωπεύουν ευρέως χρησιμοποιούμενους πόρους όπως οι σελίδες HTML και έχουν πρόσβαση μέσω ενός απλού πρωτοκόλλου HTTP. Οποιαδήποτε διαδικτυακή διεύθυνση URL καλεί ένα web API. Τα API του Web ονομάζονται συχνά REST  ή RESTful επειδή ο εκδότης των διασυνδέσεων REST δεν αποθηκεύει εσωτερικά δεδομένα μεταξύ αιτήσεων. Ως εκ τούτου, τα αιτήματα πολλών χρηστών μπορούν να αναμειχθούν όπως και στο διαδίκτυο.
\item[Program APIs] Τα API προγραμμάτων βασίζονται σε τεχνολογία κλήσης εξ αποστάσεως (Remote Procedure Call) (RPC), που καθιστά ένα εξ αποστάσεως συστατικό του προγράμματος να φαίνεται τοπικό για το υπόλοιπο λογισμικό. Για παράδειγμα η σειρά API της Microsoft WS.
\end{description}

Καθώς η συγκεκριμένη εφαρμογή έχει επικοινωνία με web application έχει υλοποιηθεί ένα web API.

\section{REST - RESTful API}
Οι REST (Representational state transfer) ή RESTful υπηρεσίες, όπως η συγκεκριμένη, αποτελούν έναν τρόπο παροχής λειτουργικότητας μεταξύ συστημάτων στο Διαδίκτυο. Οι υπηρεσίες Web που είναι συμβατές με τους κανόνες που ορίζει το REST επιτρέπουν στα συστήματα να έχουν πρόσβαση και να χειρίζονται τα δεδομένα του διαδικτύου χρησιμοποιώντας ένα ομοιόμορφο και προκαθορισμένο σύνολο λειτουργιών. Υπάρχουν και άλλες μορφές υπηρεσιών με τους δικούς τους κανόνες και λειτουργίες, όπως το WSDL και το SOAP.
\par Σε μια RESTful υπηρεσία , τα αιτήματα \textit{(requests)} που υποβάλλονται στο URL ενός πόρου θα προκαλέσουν μια απάντηση \textit{(response)} που μπορεί να είναι XML, HTML, JSON ή κάποια άλλη καθορισμένη μορφή. Η απόκριση μπορεί να επιβεβαιώσει ότι έχουν γίνει ορισμένες αλλαγές στον αποθηκευμένο πόρο και μπορεί να παρέχει συνδέσεις υπερκειμένου με άλλους σχετικούς πόρους ή συλλογές πόρων. Χρησιμοποιώντας HTTP, όπως είναι πιο συνηθισμένο, το είδος των διαθέσιμων λειτουργιών περιλαμβάνει εκείνες που είναι προκαθορισμένες από τα ρήματα HTTP GET, POST, PUT, DELETE κ.τ.λ

\subsection{Χαρακτηριστικά του REST}
Οι κανόνες που ορίζονται από το REST εξυπηρετούν στην διασφάλιση ορισμένων πλεονεκτημάτων που ορίζουν τις RESTful υπηρεσίες όταν υλοποιούνται σωστά.Τα παρακάτω είναι μερικά από τα χαρακτηριστικά που προσπαθεί να διασφαλίσει το REST
\begin{itemize}
\item Performance 
\item Scalability
\item Simplicity
\item Modifiability
\item Reliability
\end{itemize}


\subsection{Από URL σε HTTP}
Ένας από τους πιο βασικούς κανόνες που ορίζει το REST είναι η αντιστοίχηση των URL με συγκεκριμένα HTTP ρήματα και λειτουργίες.
Ο παρακάτω πίνακας περιγράφει ένα από τα αντικείμενα της εφαρμογής, το \textit{job} και πως αντιστοιχίζονται τα URL της web εφαρμογής με τα HTTP calls του API.

\begin{center}
\begin{tabular}{ |c|c|c|c| } 
\hline
URL & HTTP GET & col3 \\
\hline
http://{backend}/api/jobs/ & cell5 & cell6 \\ 
2 & cell8 & cell9 \\ 
\hline
\end{tabular}
\end{center}








\include{blank_page}

\end{document}